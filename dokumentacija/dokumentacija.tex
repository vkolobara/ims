%%%%%%%%%%%%%%%%%%%%%%%%%%%%%%%%%%%%%%%%%%%%%%%%%%%%%%%%%%%%%%%%%%%%%%
% LaTeX Example: Project Report
%
% Source: http://www.howtotex.com
%
% Feel free to distribute this example, but please keep the referral
% to howtotex.com
% Date: March 2011 
% 
%%%%%%%%%%%%%%%%%%%%%%%%%%%%%%%%%%%%%%%%%%%%%%%%%%%%%%%%%%%%%%%%%%%%%%
% How to use writeLaTeX: 
%
% You edit the source code here on the left, and the preview on the
% right shows you the result within a few seconds.
%
% Bookmark this page and share the URL with your co-authors. They can
% edit at the same time!
%
% You can upload figures, bibliographies, custom classes and
% styles using the files menu.
%
% If you're new to LaTeX, the wikibook is a great place to start:
% http://en.wikibooks.org/wiki/LaTeX
%
%%%%%%%%%%%%%%%%%%%%%%%%%%%%%%%%%%%%%%%%%%%%%%%%%%%%%%%%%%%%%%%%%%%%%%
% Edit the title below to update the display in My Documents
%\title{Project Report}
%
%%% Preamble
\documentclass[paper=a4, fontsize=11pt]{scrartcl}
\usepackage[T1]{fontenc}
\usepackage[utf8]{inputenc}
\usepackage{amsmath,amsfonts,amsthm} % Math packages
\usepackage[pdftex]{graphicx}	
\usepackage{url}
\usepackage[skip=2pt]{caption} % example skip set to 2pt
\usepackage[croatian]{babel}
\usepackage{listings}
\usepackage{enumitem}

%%% Custom sectioning
\usepackage{sectsty}
\allsectionsfont{\centering \normalfont\scshape}


%%% Custom headers/footers (fancyhdr package)
\usepackage{fancyhdr}
\pagestyle{fancyplain}
\fancyhead{}											% No page header
\fancyfoot[L]{}											% Empty 
\fancyfoot[C]{}											% Empty
\fancyfoot[R]{\thepage}									% Pagenumbering
\renewcommand{\headrulewidth}{0pt}			% Remove header underlines
\renewcommand{\footrulewidth}{0pt}				% Remove footer underlines
\setlength{\headheight}{13.6pt}


%%% Equation and float numbering
\numberwithin{equation}{section}		% Equationnumbering: section.eq#
\numberwithin{figure}{section}			% Figurenumbering: section.fig#
\numberwithin{table}{section}				% Tablenumbering: section.tab#

%%% Maketitle metadata
\newcommand{\horrule}[1]{\rule{\linewidth}{#1}} 	% Horizontal rule

\title{
		%\vspace{-1in} 	
		\usefont{OT1}{bch}{b}{n}
		\normalfont \normalsize \textsc{Fakultet Elektrotehnike i Računarstva} \\ [25pt]
		\horrule{0.5pt} \\[0.4cm]
		\huge Inteligentni multiagentski sustavi:\\Dokumentacija projekta\\
		\horrule{2pt} \\[0.5cm]
}
\author{
		\normalfont 								\normalsize
        Vinko Kolobara\\[-3pt]		\normalsize
        \today
}
\date{}


%%% Begin document
\begin{document}
\maketitle

\pagebreak

\section{Uvod}

Cilj projekta bio je osmisliti i implementirati strategiju za igru Pacman, i to i za Pacman-a, a i za duhove. Pacman za cilj ima pojesti svu hranu na mapi prije smrti, a duhovima je cilj ubijati Pacman-a i spriječiti ga u njegovoj namjeri.

Za implementaciju je korišten framework koji je napravljen za potrebe predmeta. Pacman i duhovi kao agenti imaju ograničeno vidno polje, u mogućnosti su na prethodno posjećena polja ostavljati poruke drugim agentima (npr. duhovi za međusobnu komunikaciju i praćenje Pacman-a) te imaju sve potrebne informacije o vidljivim poljima (je li na polju hrana, duh, zid, pojačanje...).

\section{Opis strategije agenata}
Za oba agenta je bilo potrebno smisliti različite strategije implementirajući metodu koja u nekom stanju određuje koji sljedeći (dozvoljeni) potez će agent napraviti. Kao udaljenost se za sve agente koristi Manhattan udaljenost, s tim da se ne provjerava bi li taj put išao preko zida.

\subsection{Strategija za Pacman-a}
Pacman u svakom stanju prvo pregleda svo vidljivo polje i zapamti polja na kojim vidi hranu i pojačanje. Nakon toga odredi hranu koja mu je najbliža i nju postavi za trenutni cilj te odredi koji će ga od trenutno dozvoljenih poteza odvesti najbliže cilju. 

Ako se blizu te pozicije nalazi duh, onda će krenuti u smjeru suprotnom od duha iako će se tako udaljavati od svog cilja. U slučaju da je pokupio pojačanje, jednostavno će ignorirati duhove i ići prema ciljnom stanju. 

Također, pamti i zadnjih 5 stanja u kojima je bio i pokušat će ne doći ponovno u ta stanja ako ikako može. 

Dodatno, moguće je da Pacman nakon nekog vremena ne uspije doći do ciljnog stanja i tada se to stanje skloni iz liste ciljnih stanja i spremi u drugu listu koja pamti stanja u koja je pokušao ići, ali nije uspio. U tom slučaju, ali i u slučaju da uspije doći do ciljnog stanja, ponovno traži najbliže polje s hranom i njega postavlja kao ciljno. 

Ako lista sa ciljnim stanjima ostane prazna, a nije kraj igre, to znači da u neuspjelim stanjima postoje stanja i tada bira nasumično neko od tih stanja umjesto najbližeg.

\subsection{Strategija za duhove}
Duhovi također prvo pregledaju vidljivo polje i provjeravaju postoji li na nekom od tih stanja zapis da je Pacman tu bio ili da je Pacman trenutno tu. Ako je to istina, a i ako Pacman nema pojačanje, to stanje se postavlja kao ciljno i određuje se sljedeći potez koji će ga dovesti najbliže cilju. 

Ako Pacman ima pojačanje i duh je blizu polja na kojem se Pacman nalazi, duh će krenuti u suprotnom smjeru od Pacman-a. Tako duh još uvijek može pratiti Pacman-a, ali sa sigurne udaljenosti da ne bude pojeden. 

Ako je duh vidio Pacman-a, na to stanje zapisuje da je vidio Pacman-a kako bi drugi duhovi znali. Nakon nekog vremena, ta informacija se ukloni (ako su duhovi dovoljno puta vidjeli poruku). 

Ako duh ne vidi Pacman-a (ili ga izgubi iz vidnog polja), a ni nema polja na kojem piše da je Pacman tu bio, onda duh bira nasumični smjer, s tim da pamti prethodnih 15 stanja u kojima je bio i trudi se ne ići u ta stanja. Tako je omogućeno istraživanje prostora duhovima.

\section{Zaključak}
Kao konačni rezultat, implementirana je relativno uspješna strategija za obje vrste agenata. U ovoj implementaciji, Pacman je uglavnom gubitnik. Razlog tome je što on ima dosta složeniji zadatak od duhova, mora tražiti hranu, najbolji put do hrane, izbjegavati duhove i sl., dok duhovi za cilj imaju samo pojesti Pacman-a i istraživati stanja dok ga ne vide.

\end{document}